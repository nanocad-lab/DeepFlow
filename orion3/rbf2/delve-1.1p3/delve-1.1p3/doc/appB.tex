%
% $Id: appB.tex,v 1.4.2.2 1996/06/14 02:37:51 carl Exp $
%
\newpage

\section{CONTRIBUTING TO THE DELVE ARCHIVE}\label{app-submit}
\thispagestyle{plain}
\setcounter{figure}{0}
\chead[\fancyplain{}{\thesection.\ CONTRIBUTING TO THE DELVE ARCHIVE}]
      {\fancyplain{}{\thesection.\ CONTRIBUTING TO THE DELVE ARCHIVE}}

The ultimate aim of the \delve{} project is to collect datasets,
implementations of learning methods, and the results of learning
experiments from a wide variety of sources. If you have datasets,
methods or results which might be of interest to other users you can
submit these to the \delve{} archive. To make contributions, you can
put files on our ftp-server \texttt{ftp.cs.utoronto.ca} in the directory
\texttt{/pub/incoming}, and notify us by email to 
\texttt{delve@cs.utoronto.ca}. The submitted files should preferably
conform to the usual \delve{} conventions, and be in the form of a
compressed \texttt{tar} file.

We welcome contributions of datasets to \delve{}. We are particularly
seeking large real-world datasets, and realistic simulation programs
that can be used to create large datasets. Contributions of datasets
should be accompanied by descriptions of the data. For real datasets
both the data in its original form and in \delve{} format should be
supplied, as well as descriptions of the relevant context and the
attributes recorded. Also suggestions for prototasks together with
specifications of prior information should be included. Naturally,
proprietary data cannot be included in \delve{} without permissions.
For simulated and artificial datasets, programs to generate the data
should be supplied (if possible) as well as descriptions of the data
attributes and suggestions for prototasks and priors, etc.

You may also contribute new learning methods to the archive.
Typically, you would also provide results of running your method on
various \delve{} datasets. You can conveniently submit the whole
methods directory pertaining to your method. Also you need to supply a
detailed description of your method. Remember, that the description
should be detailed enough that someone else can re-implement the
method and get comparable results to the ones you might get for any
dataset to which the method is applicable.  The easiest way to attain
this, is if your method is fully automatic. In particular, you should
make sure that your description includes:

\begin{itemize}
\item implemetational details allowing someone else to re-implemet your
method with similar results
\item discussion of the role of all parameters of the method
\item discussion of the heuristic rules for setting all parameters of the
method on the basis of a particular application, including convergence
criteria for iterative methods
\item detailed discussion of how attributes should be encoded for the
method
\end{itemize} 

Finally, it would be convenient if source code of the program
implementing you method can be included in \delve{}. This may help
clarify details of the implementation, help other researches to
easily use the methods and help with identifying possible
bugs. Authors should take care not to submit implementations
containing any parts whose copyrights prohibit public distribution.

For all contributions it should be considered that submission to
\delve{} is a form of publication, and once contributions are released
with \delve{} they cannot in general be retracted, since other people
may have used them in their research. Therefore, care should be taken
to avoid submissions of erroneous material. If a bug should be
discovered in a learning method a new corrected version can be
submitted under a different name; eg.~a buggy version of {\tt loess-1}
could be succeeded by a corrected version named {\tt loess-2} --- but
the original method and its results would be retained in the archive.

You may also submit experimental results using new combinations of
methods and datasets that are already in \delve{}. If you repeat
experiments for which results are already in the archive, it is of
interest whether your results were comparable to the earlier results.
Notes of such confirmations can be included in \delve{}, but for
practical reasons only one set of results can be maintained for each
method.

All submitted material will be presented in \delve{} with the date,
name and address (or email) of the contributor(s) allowing further
clarifications and collaboration.
